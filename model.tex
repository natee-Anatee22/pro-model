\documentclass{report}

\usepackage{amsmath, amssymb}
\usepackage{amsthm}
\usepackage[thaifont=AngsanaNew]{thaispec}

\usepackage{biblatex}
\addbibresource{AA.bib}



\title{ความหมายของ \textbf{codes} ในโปรแกรม \textbf{python}\\ทางคณิตศาสตร์}
\author{นางสาวอนาตี มะหะหมัด}
\date{1 มีนาคม 2567}

\begin{document}
\maketitle
\chapter{ที่มาและความสำคัญของโปรแกรม python}
\section{ที่มา}
Python ถูกคิดค้นครั้งแรกในปี 1989 โดย Guido van Rossum ชาวเนเธอร์แลนด์
ซึ่งสำเร็จการศึกษาระดับปริญญาโทสาขาคณิตศาสตร์และวิทยาการคอมพิวเตอร์จาก University of Amsterdam
ประเทศเนเธอร์แลนด์ เขาเคยได้เข้าร่วมทำงานกับ Centrum
Wiskunde and Informatica (CWI) ประเทศเนเธอแลนด์,
National Institute of Standards and Technology
(NIST) และ Corporation for National Research Initiatives (CNRI) ประเทศสหรัฐอเมริกา และได้ร่วมงานกับบริษัท Google ตั้งแต่ปี2005 จนถึงปี2012 และย้ายไปทำงาน
ที่ Dropbox ในปี2013 จนกระทั่งเกษียณตัวเองในเดือนตุลาคม
ปี 2019 และได้กลับเข้ามาเริ่มงานอีกครั้งกับ Microsoft เมื่อ
ปลายปี2020 โดยมีเป้าหมายหลักคือคือการพัฒนาให้Python
ทำงานได้เร็วขึ้นอย่างน้อย 5 เท่าภายในระยะเวลา 4 ปี ทั้งนี้ที่มาของชื่อภาษาโปรแกรมไพธอน มา
จากตอนที่ Van Rossum กำลังเริ่มพัฒนาภาษาโปรแกรมไพธอน เขาได้อ่านสคริปต์ของรายการทีวี
“Monty Python’s Flying Circus” ซึ่งเป็นซีรีส์ตลกของช่อง BBC ในยุค 1970s ทำให้เขาได้คิดว่า
เขาต้องการชื่อที่สั้น มีเอกลักษณ์ และค่อนข้างลึกลับน่าค้นหา เขาจึงได้ตัดสินใจใช้ชื่อ Python สำหรับ
ภาษาโปรแกรมที่เขากำลังพัฒนา
ภาษาโปรแกรมไพธอนเป็นภาษาระดับสูงที่ได้รับการพัฒนาจากการผสมผสานความหลากหลายของ
ภาษาอื่น ๆ เช่น ABC, Modula-3, Icon, Perl, Lisp, Smalltalk เป็นต้น นอกจากนี้ยังมีความสามารถ
ในการจัดการหน่วยความจำแบบอัตโนมัติ รวมไปถึงการจัดการในเรื่องของตัวแปรที่สร้างขึ้นมาใช้งาน
โดยไม่ต้องกำหนดชนิดข้อมูล
เขาจึงพยายามสร้างภาษาขึ้นมาใหม่ที่เข้าใจง่ายและไม่ซับซ้อน จึงเป็นจุดเริ่มต้นให้เขาคิดค้นภาษา Python ขึ้นมานั่นเอง
โดย Guido ดัดแปลงรูปแบบภาษาบางอย่างมาจากภาษา ABC มาพัฒนาลงในภาษา Python เขาเริ่มเผยแพร่ Python 1.0 เวอร์ชันแรกในปี 1994 ซึ่งเป็นภาษาที่มีอายุมากกว่าภาษา Java ที่เกิดขึ้นในเวลาต่อมาในปี 1996\cite{OO} 
\section{ความสำคัญของโปรแกรมpython}
Python มีประโยชน์สำหรับการเขียนโค้ดฝั่งเซิร์ฟเวอร์ เป็นภาษาโปรแกรมพื้นฐานที่นำไปต่อยอดได้หลายรูปแบบ เรียกว่าอยู่ที่จะใช้ทำอะไรมากกว่า เนื่องจากมีความยืดหยุ่นคล่องตัวสูง ทั้งยังมี Tools และ Library Support ฟรีเยอะ หาข้อมูลได้ง่าย แต่ที่นิยมนำไปใช้งานอย่างแพร่หลายเนื่องจากมีไลบรารีจำนวนมากที่ประกอบด้วยโค้ดที่เขียนไว้ล่วงหน้าสำหรับฟังก์ชันแบ็คเอนด์ที่ซับซ้อน นักพัฒนายังใช้เฟรมเวิร์ก Python ที่หลากหลายซึ่งมีเครื่องมือที่จำเป็นทั้งหมดเพื่อสร้างเว็บแอปพลิเคชันได้เร็วขึ้นและง่ายขึ้นอีกด้วย ตัวอย่างเช่น นักพัฒนาสามารถสร้างโครงสร้างเว็บแอปพลิเคชันได้ภายในไม่กี่วินาที เนื่องจากไม่จำเป็นต้องเขียนขึ้นใหม่ทั้งหมด จากนั้นนักพัฒนาสามารถทดสอบได้โดยใช้เครื่องมือทดสอบของเฟรมเวิร์ก โดยไม่ต้องพึ่งพาเครื่องมือทดสอบภายนอก\cite{BB}

\chapter{codes ที่มีความสัมพันธ์ที่จะสามารถไปสร้างแบบจำลองทางคณิตศาสตร์}
\begin{enumerate}
	\item{คำสั่ง abs() ใช้สำหรับหาค่าจำนวนเต็มบวก}
	\item{คำสั่ง float()ใช้ในกรณีที่ต้องการแปลงชนิดข้อมูลจำนวนเต็ม ให้เป็นชนิดข้อมูลจำนวนทศนิยม}
	\item{คำสั่ง complex()ใช้ในกรณีที่ต้องการแปลงเป็นชนิดข้อมูลจำนวนเชิงซ้อนให้เรียกใช้งาน}
	\item{คำสั่ง int()ใช้ในกรณีที่ต้องการแปลงเป็นชนิดข้อมูลจำนวนเต็มให้เรียกใช้งาน}
	\item{คำสั่ง math.modf(x) แยกจำนวนเต็มและเศษของจำนวนจริงออกจากกันx คือ จำนวนจริง}
	\item{คำสั่ง isalnum() คืนค่า True ถ้าสตริงประกอบด้วยตัวอักษรปนกับตัวเลข หรือแค่ตัวอักษรตัวเลขอย่างใดอย่างหนึ่งเพียงอย่างเดียวถ้ามีสัญลักษณ์อื่น หรือช่องว่างจะคืนค่าFalse}
	\item{คำสั่ง isalpha() คืนค่า True ถ้าสตริงประกอบด้วยตัวอักษรทั้งหมด ถ้ามีสัญลักษณ์ ตัวเลข หรือช่องว่างจะคืนค่า False}
	\item{คำสั่ง isdigit() คืนค่า True ถ้าสตริงประกอบด้วยตัวเลขทั้งหมด ถ้ามีตัวอักษร สัญลักษณ์ หรือช่องว่าง จะคืนค่า False}
	\item{คำสั่ง isdecimal() ใช้สำหรับตรวจสอบตัวเลขแสดงผลลัพธ์เหมือนเมธอดisdigit()}
	\item{คำสั่ง isnumber() ใช้สำหรับตรวจสอบตัวเลข แสดงผลลัพธ์เหมือนเมธอดisdigit()}
	\item{คำสั่ง append() เพิ่มข้อมูลต่อท้ายลิสต}
	\item{คำสั่ง len() ฟังก์ชันที่ใช้แสดงจำนวนข้อมูลที่มีอยู่ในลิสต์}
	\item{คำสั่ง math.sqrt(x) ค่ารากที่สองของ x x คือ จำนวนจริง ที่ไม่ต่ำกว่า 0}
	\item{คำสั่ง math.copysign(x,y) ส่งกลับค่าของ x เป็นจำนวนจริงและใส่เครื่องหมายตามค่าของ yxคือ จำนวนเต็มหรือจำนวนจริงy คือ จำนวนเต็มหรือจำนวนจริง}
	\item{คำสั่ง divmod() จะคืนค่าผลหารและเศษจากการหารออกมา}
	\item{คำสั่ง enumerate() ใช้สำหรับกำหนดหมายเลขแบบเรียงลำดับ}
	\item{คำสั่ง range() ใช้สำหรับสร้างกลุ่มข้อมูลตัวเลขจำนวนเต็ม}
	\item{คำสั่ง round() ใช้สำหรับปัดเศษจำนวนทศนิยม}
	\item{คำสั่ง math.exp(x) ค่า e ยกกำลัง x}
	\item{คำสั่ง math.floor(x) หาจำนวนเต็มที่มีค่ามากที่สุด}
	\item{คำสั่ง math.factorial(x) หาค่า factorialของ x}
	\item{คำสั่ง math.gcd(x,y) หาค่าตัวหารร่วมมากที่มีค่ามากที่สุด}
	\item{คำสั่ง math.fmod(x,y) หาค่าเศษจากการหาร}
	\item{คำสั่ง math.fsum(x) หาค่าผลรวมจากชนิดข้อมูลแบบเรียงลำดับ}
\end{enumerate}
\printbibliography

\end{document}